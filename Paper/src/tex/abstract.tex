\vspace*{2cm}
% Abstract
{\bf\Large Abstract} \\ [1em]
The article discusses hospital information systems (HIS) as socio-technical systems that consist of technical components and human aspects like hospital staff and patients. It analyses HIS on three layers: tasks and entity types, application components, and physical data processing components. The article provides examples of common application components, such as patient administration systems, medical documentation systems, and radiology information systems, and explains their tasks and functions. The document includes several exercises related to HIS modeling, such as assigning objects to different classes and distinguishing the fundamental components of HIS. The primary management interest in HIS is in information processing, called information management. It also covers the three perspectives of information management: strategic, tactical, and operational. The article emphasizes the importance of managing the inter-layer relationships between these components for a reliable HIS. Exercises and challenges are provided at the end of the article. A digital patient board system aims to enhance access to patient information for healthcare providers and patients. The approach to ensuring control over access to confidential patient information is based on the expected relationships between staff and patients, identifying roles and corresponding rights to limit access. Also, maintain an audit trail of all occasions when a patient's record is accessed. The system provides an open environment for healthcare providers to design and deploy services, improving communication possibilities between healthcare providers and patients.

\cleardoublepage
