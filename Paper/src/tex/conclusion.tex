\chapter{Related Works}
For related works, four papers are explored that either talked about Hospital Information Systems or Modelling various aspects of the Hospital Information System. 

The papers have approached the modeling of the E-health system using UML diagrams such as use case diagrams, activity diagrams, class diagrams, and component diagrams. The authors have identified several processes involved in clinical health services, including registration, polyclinic process, medicine, recipe, and doctor's schedule. They have integrated these processes to ensure that every patient who takes a medical check can register with only one registration and wait until called. 

The authors of the paper \textit{E-health as a Service Software of Medical System in UML
Modeling} have also identified some problems in the existing system, such as patients needing to be recorded in the insurance program provided by the government, causing trouble when searching for data. They have proposed a prototype modeling approach to construct the software, which is appropriate and straightforward. The prototype modeling approach allows the software to be delivered early, so the client can see the software and supervise the building of the software directly. 

They have also developed a component diagram that shows the application link among components built in the PHP language. They have created a class diagram that describes the system's structure, with classes having attributes, methods, and operations/methods. The classes in the system structure must be able to perform the functions in accordance with the system's needs. \cite{EHS}

The approach taken by the paper \textit{Health Information Systems} is to provide a comprehensive understanding of the management perspectives and inter-layer relationships of Hospital Information Systems (HIS). The authors propose three management perspectives: strategic, tactical, and operational, and describe their tasks and responsibilities. They also discuss the interlayer relationships between tasks and entity types, application components, and physical data processing components. The paper also introduces the PDCA (\textit{plan-do-check-act}) cycle and project management tools for organizing work in different phases. The result of the paper is a systematic approach to analyzing HIS, which can help hospital managers to make informed decisions about HIS implementation, operation, and maintenance.\cite{HIS}

The paper \textit{Modeling hospital information systems - Part 1: The revised three-layer graph-based meta model 3LGM2} presents a meta-model called 3LGM2 (\textit{Three Layer Graph Based Meta Model}) for modeling hospital information systems (HIS) using concepts on three layers: domain layer, logical tool layer, and physical tool layer. The approach is based on a case study that identified requirements for modeling HIS, and the meta-model was designed to meet those requirements. The paper also describes a software tool that supports the creation of 3LGM2 compliant models in a graphical way and can detect shortcomings at the logical or physical tool layers that make it impossible to satisfy the information needs at the domain layer.

The 3LGM2 meta-model is represented using the Unified Modeling Language and combines a functional meta-model with technical meta-models. It distinguishes between three different layers within an information system. It replaces the former 'procedure level' with the domain layer, which describes a hospital independently of its implementation by its enterprise functions. The logical tool layer focuses on application components, and the physical tool layer describes physical data processing components.

Compared to UML, the 3LGM2 approach has a lot of interlayer relationships and can be used as an ontology for describing HIS in natural language. However, the paper acknowledges some problems, such as the need for more process modeling and the considerable effort to model even a small information system from scratch. The paper concludes that reliable methods and tools can support hospital strategic information management. However, further evaluations on different sites are needed to show whether building and keeping up-to-date large models is possible.\cite{MHIS}

The paper \textit{UML-based ontology for describing hospital information system architectures} presents an ontology-based approach for describing hospital information systems architectures that allow the description of the essential information system components, their relationships, and their dependencies. The approach aims to answer relevant information management questions and provide criteria for the quality of a hospital information system. The ontology is presented using the Unified Modeling Language, and the examples are shown as intuitive graphical notations. 

The paper differs from other information systems planning and modeling approaches that place modeling aspects in the foreground, whereas interpretation and evaluation aspects are neglected. The presented ontology's selection of concepts for describing hospital information systems architectures and the relationships among these concepts is oriented towards information management questions that must be answered and information management tasks that must be supported in daily work. In comparison, the Unified Modeling Language is a general-purpose modeling language used in software engineering for visualizing, specifying, constructing, and documenting the artifacts of a software system. UML is not specific to hospital information systems but can be used to model any software system. The paper's approach is specific to hospital information systems and provides criteria for the quality of such systems.\cite{UHIS}

\let\cleardoublepage\clearpage
\chapter{Conclusion}
In conclusion, this research paper has presented a comprehensive analysis of modeling hospital information systems. By examining the various components of HIS, including electronic health records, and hospital billing and administrative systems, we have highlighted the critical role of modeling in designing and implementing effective healthcare information systems.

Through the utilization of modeling concepts such as Monticore and Business Process Model and Notation (BPMN), this paper has provided insights into the application of these techniques in capturing the structure, behavior, and interactions within hospital information systems. The use of Monticore has facilitated the creation of models such as class diagrams, use case diagrams, and sequence diagrams, while BPMN diagrams have enabled the representation of complex workflows and processes within the healthcare organization.

One of the notable strengths of the approach presented is the potential for reusability. The modeling techniques and concepts explored can serve as a foundation for future HIS projects, allowing stakeholders to build upon the existing knowledge and experience. The use of standardized notations and languages, such as UML and BPMN, enables interoperability and collaboration among different healthcare organizations and stakeholders.

Looking ahead, future work in this domain should address emerging challenges and leverage new technologies to further enhance the modeling and implementation of hospital information systems. For instance, advancements in artificial intelligence and machine learning can be harnessed to develop more intelligent clinical decision support systems. Additionally, the integration of Internet of Things (IoT) devices and wearable technologies into HIS modeling can enable real-time monitoring of patients and facilitate data-driven decision-making.

Furthermore, it is essential to continue exploring the ethical and legal considerations surrounding patient data privacy and security. With the increasing reliance on interconnected systems and the sharing of patient data, robust measures must be implemented to protect sensitive information and ensure compliance with relevant regulations.

By continuing to advance the modeling techniques and methodologies in hospital information systems, healthcare administrators, IT professionals, and researchers can contribute to the ongoing transformation of healthcare delivery, ultimately improving patient outcomes, streamlining workflows, and enhancing the overall quality of care.
% % Bild einbinden
% \begin{figure}[ht!]
% \begin{center}\includegraphics[width=5cm]{src/pic/logo}\end{center}
% \caption{Das SE Logo}
% \label{Logo}
% \end{figure}

\let\cleardoublepage\clearpage
